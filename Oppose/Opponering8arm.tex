\documentclass{article}
\usepackage{graphicx}
\usepackage[swedish,english]{babel}
\usepackage{amsmath}
\usepackage[utf8]{inputenc}
\usepackage{booktabs}
\usepackage{graphicx}
\usepackage{epstopdf}
\usepackage{tabularx}
\usepackage{placeins}
%\usepackage[bw]{mcode}
\usepackage{listings}
\usepackage{color} %red, green, blue, yellow, cyan, magenta, black, white
\definecolor{mygreen}{RGB}{28,172,0} % color values Red, Green, Blue
\definecolor{mylilas}{RGB}{170,55,241}


\begin{document}

\title{MF123x\\ Opposition on Emil Grundén \& Krister Kölzow's 8arm \\}
\author{Alexander Ramm \& Mikael Sjöstedt}
%\subtitle{Hemuppgift 3}
\maketitle
\clearpage

\section{General content}
The authors clearly states early in the report that the  project is focused to the \textit{Open Source} community. Throughout the paper the authors explain the difficulties with the project but also instructs the reader of how a similar robot can be built, which is the main goal of the paper. All the main components are mentioned and why they are used which is essential for a DIY'er.

The report is written at a good level for the graduate student as well as the hobby'ist looking for some in-depth explanation. There is a line of argument throughout the report making it easy to read making it a good report overall. 
\\ \\
The research question is formulated as
\begin{quote}
\textit{How precise can the flow of a peristaltic pump be controlled when using only cheap and easily available components for construction}
\end{quote}
However the phrase \textit{"How precise"} is quite vague and not further explained in the scope. The conclusion does not really reconnect to the question either, only that it is possible to "get a good precision from it" which does not answer the question.  
\\ \\
The theory chapter covers the most crucial theory but is more aimed towards the DIY'er. For an engineer at an graduate level, the information is redundant. No real in-depth theory is explained, making some of the subsections in the theory chapter unnecessary.
\\ \\
The error of the pump is specified for 5\% to reach 1 cl of liquid. The scale used to determine how much liquid pumped through the system was an ordinary kitchen scale with a precision of one gram. How come the pump error of 5\% is lower than the error of the scale of 10 \% considering a centilitre of water? 
\\ \\
Maybe a lot of emphasising words that makes the paper "friendly" and easy to read but also reduce the scientific worth.


\section{Notes}

\begin{itemize}
\item The research question is formulated as "How precise...", define what this implicates

\item Wikipedia source

\item Why does a smaller motor increase pump-time? Can't it revolve faster? 

\item Why not use a stepper motor if the number of revolutions is specified for every amount of liquid?

\item If another of the same type were used, could the precision be increased?

\item How have the results been concluded? With several measurements, it isn't specified. How come the error is smaller than the measurement equipment tolerance. 

\item Papernotes
	
\end{itemize}



\end{document}